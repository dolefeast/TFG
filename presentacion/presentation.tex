%%%%%%%%%%%%%%%%%%%%%%%%%%%%%%%%%%%%%%%%%%%%%%%%%%%%%%%%%%%%%%%%%%%%%%%%%%
%%%%%% Ajit Kumar Sahoo, School of Computer and Information Sciences,
%%%%%% University of Hyderabad, Hyderabad, India-500046
%%%%%% Email: sahooajitkumar85@gmail.com
%%%%%% https://ajitsahoocs.github.io/
%%%%%%%%%%%%%%%%%%%%%%%%%%%%%%%%%%%%%%%%%%%%%%%%%%%%%%%%%%%%%%%%%%%%%%%%%%%
\documentclass{beamer}
\hypersetup{pdfpagemode=FullScreen} %full screen mode
\setbeamertemplate{navigation symbols}{}
\usepackage[english]{babel} %designed for typesetting lebels(addr lebel,sticky lebels,etc)
\usepackage[latin1]{inputenc} %Accept different input encodings
\usepackage{amsmath,amssymb} 
\usepackage[T1]{fontenc} %for font encoding
\usepackage{times} % use times font instead of default
\usepackage{curves} %for drawing curves
\usepackage{verbatim} %paragraph making environment 
\usepackage{multimedia} %for multimedia like animation,movie etc...
\usepackage{mathptmx} % font style
\usepackage{graphicx} % Allows including images
\usepackage{booktabs} % Allows the use of \toprule, \midrule and \bottomrule in tables
\usepackage{hyperref}
\usepackage{xcolor}

\usepackage{algorithm,algorithmic}
\renewcommand{\algorithmicrequire}{\textbf{Input:}}
\renewcommand{\algorithmicensure}{\textbf{Output:}}
\usepackage{lipsum}
\setbeamertemplate{caption}[numbered] % For numbering figures



\mode<presentation> {

% The Beamer class comes with a number of default slide themes
% which change the colors and layouts of slides. Below this is a list
% of all the themes, uncomment each in turn to see what they look like.

%\usetheme{default}
%\usetheme{AnnArbor}
%\usetheme{Antibes}
%\usetheme{Bergen}
%\usetheme{Berkeley}
%\usetheme{Berlin}
%\usetheme{Boadilla}
%\usetheme{CambridgeUS}
%\usetheme{Copenhagen}
%\usetheme{Darmstadt}
%\usetheme{Dresden}
%\usetheme{Frankfurt}
%\usetheme{Goettingen}
%\usetheme{Hannover}
%\usetheme{Ilmenau}
%\usetheme{JuanLesPins}
%\usetheme{Luebeck}
\usetheme{Madrid}
%\usetheme{Malmoe}
%\usetheme{Marburg}
%\usetheme{Montpellier}
%\usetheme{PaloAlto}
%\usetheme{Pittsburgh}
%\usetheme{Rochester}
%\usetheme{Singapore}
%\usetheme{Szeged}
%\usetheme{Warsaw}

% As well as themes, the Beamer class has a number of color themes
% for any slide theme. Uncomment each of these in turn to see how it
% changes the colors of your current slide theme.

%\usecolortheme{albatross}
%\usecolortheme{beaver}%this one also
%\usecolortheme{beetle}
%\usecolortheme{crane} % also use this one
%\usecolortheme{dolphin}
%\usecolortheme{dove}
%\usecolortheme{fly}
%\usecolortheme{lily}
%\usecolortheme{orchid}
%\usecolortheme{rose}
%\usecolortheme{seagull}
%\usecolortheme{seahorse}
\usecolortheme{whale} % Best one
%\usecolortheme{wolverine} %use can use this also

%\setbeamertemplate{footline} % To remove the footer line in all slides uncomment this line
%\setbeamertemplate{footline}[page number] % To replace the footer line in all slides with a simple slide count uncomment this line

%\setbeamertemplate{navigation symbols}{} % To remove the navigation symbols from the bottom of all slides uncomment this line
}

\usepackage{graphicx} % Allows including images
\usepackage{booktabs} % Allows the use of \toprule, \midrule and \bottomrule in tables
\setbeamercovered{transparent}
\setbeamertemplate{bibliography item}[text]
\setbeamertemplate{theorems}[numbered]
\setbeamerfont{title}{size=\Large}%\miniscule,command,tiny, scriptsize,footnotesize,small,normalsize,large,Large,LARGE,huge,Huge,HUGE
\setbeamerfont{date}{size=\tiny}%{\fontsize{40}{48} \selectfont Text}

\setbeamertemplate{itemize items}[ball] % if you want a ball
\setbeamertemplate{itemize subitem}[circle] % if you wnat a circle
\setbeamertemplate{itemize subsubitem}[triangle] % if you want a triangle


%------------------customized frame----------------------------
\newcounter{cont}

\makeatletter
%allowframebreaks numbering in the title
\setbeamertemplate{frametitle continuation}{%
   % \setcounter{cont}{\beamer@endpageofframe}%
    %\addtocounter{cont}{1}%
   % \addtocounter{cont}{-\beamer@startpageofframe}%
   % (\insertcontinuationcount/\arabic{cont})%
}
\makeatother
%-----------------------customized-----------------------------




%----------------------------------------------------------------------------------------
%	TITLE PAGE
%----------------------------------------------------------------------------------------

\title[Non-Flat universe BAO]{Baryon acoustic oscillations in a non-flat universe} % The short title appears at the bottom of every slide, the full title is only on the title page

%\author[AKS]{Ajit Kumar Sahoo \texorpdfstring{\scriptsize Regd No: 16MCPC03}{}}
\author[SSW]{\texorpdfstring{Santiago Sanz Wuhl}{Author}}

\institute[UCO] % Your institution as it will appear on the bottom of every slide, may be shorthand to save space
{

{\small Supervisors}\\
\medskip
{\normalsize Antonio J. Sarsa Rubio}\\
{\normalsize Antonio J. Cuesta V\'azquez}\\
\begin{center}
\includegraphics[width=0.2\textwidth]{uoh.png}
\end{center}
\medskip
Facultad de Ciencias \\
Universidad de C\'ordoba

%\medskip
%\textit{john@smith.com} % Your email address
}
%This will place the image at position "30 right/left and 120 up/down" relative to the top left corner of the current page.
%\titlegraphic{%
%  \begin{picture}(0,0)
%    \put(30,125){\makebox(0,0)[rt]{\includegraphics[width=2.5cm]{uoh.png}}}
%  \end{picture}}
 \date{\today} % Date, can be changed to a custom date
\begin{document}

\begin{frame}
\titlepage % Print the title page as the first slide
\end{frame}
\begin{frame}
  \frametitle{Contents}
 % \tableofcontents[pausesections,shaded]
 \tableofcontents
\end{frame}
% TABLE OF CONTENTS AT BEGIN OF EACH SECTION
\AtBeginSection[]{
  \begin{frame}<beamer>
    \frametitle{Current Section}
   \tableofcontents[currentsection]
  \end{frame}}
%----------------------------------------------------------------------------------------
%	PRESENTATION SLIDES
%----------------------------------------------------------------------------------------


%-------------------------------------------------------------------------------
%-------------------------------INTRODUCTION--------------------------------
%-----------------------------------------------------------


\section{Abstract} 
\begin{frame}[allowframebreaks]
\frametitle{Abstract}
In this Bachelor's Thesis we make use of high performance computing and data analysis tools to study the effects of slight variations in the Standard Cosmological model, the $\Lambda$ Cold Dark Matter ($\Lambda$CDM) model. This model assumes a spatially flat universe, though the observations are compatible with a nonzero value of the curvature parameter $\Omega_k$.

This work is based off the Baryon Acoustic Oscillations, a phenomenon that allows us to study the behavior of the universe in its earliest stages (the first 380.000 of its 13.8 billion years of lifetime -- a 0.003\% of the Universe's lifetime!). These oscillations shape the large scale structure of the universe, and more importantly, set a `cosmic ruler' $r_d$ with respect to which is used to measure cosmological distances, such as the Hubble distance $D_H$ and angular diameter distance $D_M$.\\

After analyzing the \textit{extended Baryon Oscillation Spectroscopic Survey }galaxy catalogue, we achieve the following results: $D_H/r_d = 18.66\pm 0.72$ y $D_M/r_d = 18.28\pm 0.53$ for a flat universe, in concordance to the results for other nonzero values of the curvature parameter $\Omega_k$ (up to a 20\% of the total density), and more importantly with previous results in the field.

\end{frame}


\section{Introduction}
\begin{frame}[allowframebreaks]
\frametitle{Big Bang?}
\end{frame}

\begin{frame}[allowframebreaks]
\frametitle{BAO}
\end{frame}


\begin{frame}[allowframebreaks]
\frametitle{$\Lambda$CDM}
\end{frame}




%-------------------------------------------------------------------------------------------------------------------------------
\section{Objectives}
\begin{frame}[allowframebreaks]
\frametitle{Objectives}
\end{frame}

\section{Materials and methods}
\begin{frame}[allowframebreaks]
\frametitle{Mathematics}
\end{frame}



\begin{frame}{Software and Hardware}

\end{frame}

\section{Results}
\begin{frame}{Results}
	 
\end{frame}
\begin{frame}{Results}
	 
\end{frame}
\begin{frame}{Results}
	 
\end{frame}

\begin{frame}{Conclusions}
\end{frame}



%\section{Information processing  issues}
%-----------------------------------------------
%\begin{frame}
%\frametitle{Information processing issues}
%\begin{itemize} 
%\item In collaborative processing
%\begin{itemize}
%\item Target detection, localization, communication, storage, querying...
%\end{itemize}
%\color{lightgray}
%\item  In networking
%\begin{itemize}
%\color{lightgray}
%\item  aggregation, routing, Data naming,...
%\end{itemize}
%\end{itemize}
%\end{frame}
%-----------------------------------------------
%\begin{frame}[fragile] % Need to use the fragile option when verbatim is used in the slide
%\frametitle{Citation}
%An example of the \verb|\cite| command to cite within the presentation:\\~
%Citation \cite{Souza} requires .
%\end{frame}
%------------------------------------------------
\section{References}
\begin{frame}[allowframebreaks]
\frametitle{References}
 \nocite{*}
 \bibliographystyle{unsrt}
\bibliography{References}
\end{frame}

%-----------------------------------------------------------------------------------------
\section{Acknowledgement}
\begin{frame} [allowframebreaks]
\frametitle{Acknowledgement}
\lipsum[1]


\end{frame}
%-----------------------------------------------------------------------------------------

\begin{frame}
\Huge{\centerline{Thank You}}
\end{frame}

%----------------------------------------------------------------------------------------

\end{document}
