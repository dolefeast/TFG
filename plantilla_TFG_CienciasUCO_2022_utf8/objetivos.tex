\chapter{Objectives}
The general objective of this final degree work is to study the behavior of baryon acoustic oscillation (BAO) observables, namely the sound horizon distance $r_s$, the Hubble parameter distance $D_H$, and the angular diameter distance $D_A$, as a function of the curvature parameter $\Omega_k$. To achieve this objective, the following specific objectives are proposed:

\begin{enumerate}
  \item Review the theoretical background of BAO observables, including their physical origin and mathematical formulation, and become familiar with the software tools Rustico, BRASS, and Python for data analysis and visualization.
  \item Learn to use these software tools for data preprocessing, analysis, and visualization of BAO-related cosmological data sets, particularly those related to the curvature parameter $\Omega_k$.
  \item Investigate the impact of different values of $\Omega_k$ on the behavior of BAO observables, and determine how this affects our understanding of the expansion history of the universe.
  \item Analyze the most recent observational data on BAO observables, obtained from experiments such as SDSS, BOSS, and eBOSS, and compare the results with theoretical predictions for different values of $\Omega_k$.
  \item Evaluate the strengths and limitations of using BAO observables to constrain the value of $\Omega_k$ and its impact on other cosmological parameters.
  \item Specific data analysis software development.
  \item Learn to control computer clusters via ssh.
\end{enumerate}
Together, these objectives will provide a comprehensive understanding of the behavior of BAO observables for different values of $\Omega_k$, and their role in constraining the curvature parameter and other cosmological parameters.




