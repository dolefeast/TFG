\chapter{Objectives}
\label{cha:objectives}
The main objectives of this work are as follows

\begin{enumerate}
  \item Review the theoretical background of BAO observables, including their physical origin and mathematical formulation, and become familiar with the software tools Rustico, BRASS, and Python for data analysis and visualization.
  \item Learn to use these software tools for data preprocessing, analysis, and visualization of BAO-related cosmological data sets, particularly those related to the curvature parameter $\Omega_k$.
  \item Investigate the impact of different values of $\Omega_k$ on the behavior of BAO observables.
  \item Analyze the most recent observational data on BAO observables, obtained from experiments such as SDSS, BOSS, and eBOSS, and compare the results with theoretical predictions for different values of $\Omega_k$.
	\item Make use of high performance computing to solve Physics problems.
  \item Specific data analysis software development.
  \item Learn to control computer clusters via SSH (Secure Shell).
\end{enumerate}
Together, these objectives will provide a comprehensive understanding of the behavior of BAO observables for different values of $\Omega_k$, and their role in constraining the curvature parameter and other cosmological parameters.




