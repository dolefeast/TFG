\chapter*{Conclusiones}
\addcontentsline{toc}{chapter}{Conclusiones}


\chapter*{Conclusions}
\addcontentsline{toc}{chapter}{Conclusions}

Relative to the objectives stated in the chapter~\ref{cha:objectives}, and seeing the results obtained in the chapter~\ref{cha:results}, the following conclusions can be made from this work.

\begin{enumerate}
	\item We have reviewed and implemented the Baryon ACoustic Oscillation (BAO) methodology to measure cosmological distances in the Universe, and we have applied it to a sample of galaxies from the extended Baryon Oscillation Spectroscopic Survey (eBOSS)
	\item We have used advanced cosmology software such as CLASS, RUSTICO and BRASS to develop a pipeline that obtains cosmological distance measurements of a galaxy catalogue using the BAO technique.
	\item In addition, we have developed code usint the Matplotlib library written in Python to generate quality level visualization of our results.
	\item The high performance computers of the University of Cordoba FQM-378 research group have been used to calculate the Fast Fourier Transform of the 2-point correlation function of eBOSS galaxies, and to determine the positions of the BAO features in the power spectrum, together with their uncertainties.
	\item Assuming a flat cosmological model, our inferred distance measurements to eBOSS galaxies (normalized to the sound horizon scale) are: $D_H/r_d = 18.68 \pm 0.72$ and $D_M/r_d = 18.13 \pm 0.58$ for the angular diameter distance, consistent with previous works~\cite{hector}.
	\item Our main result is that there is no significant dependence of these observables with respect to changes in the assumed value of the $\Omega_k$ parameter in the range of study.
\end{enumerate}

