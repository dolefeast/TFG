\chapter*{Conclusiones}
\addcontentsline{toc}{chapter}{Conclusiones}

En relación a los objetivos planteados en el capítulo \ref{cha:objectives}, y viendo los resultados obtenidos en el capítulo \ref{cha:results}, se pueden extraer las siguientes conclusiones de este trabajo.
\begin{itemize}
	\item Siendo todos lo observables BAO compatibles el uno con el otro, se puede confirmar que la metodología seguida ha sido la correcta.
	\item Los observable obtenidos son toods obtenidos con otros resultados, como por ejemplo $D_H/r_d   = 19.77\pm 0.47$, y $D_M /r_d   = 17.65\pm 0.30$~\cite{Gil_Mar_n_2020}, que asume el modelo estándar $\Lambda$CDM.
	\item Se nota también que no hay una dependencia significativa de estos observables con el parámetro de curvatura $\Omega_k$ en el intervalo estudiado. 
\end{itemize}


\chapter*{Conclusions}
\addcontentsline{toc}{chapter}{Conclusions}

Relative to the objectives stated in the chapter (ref objective chapter), and seeing the results obtained in the chapter (ref result chapter), the following conclusions can be made from this work.

\begin{itemize}
	\item Since all the obtained BAO observables were compatible with one another, the methodology followed in this work was the correct one. 
	\item The obtained observables are all compatible with other results, such as $D_H/r_d =19.77\pm 0.47$, $D_M /r_d=17.65\pm 0.30$ \cite{Gil_Mar_n_2020}, which asume the $\Lambda$CDM standard model.
	\item It was also noted that there is no significative dependecy of these observables with respect to changes in the parameter $\Omega_k$ in the studied interval.
\end{itemize}

