\chapter*{Conclusiones}
\addcontentsline{toc}{chapter}{Conclusiones}

En relación a los objetivos establecidos en el capítulo~\ref{cha:objectives}, y considerando los resultados obtenidos en el capítulo~\ref{cha:results}, se pueden extraer las siguientes conclusiones de este trabajo:

\begin{enumerate}
\item Hemos revisado e implementado la metodología de Oscilaciones Acústicas de Bariones (BAO, por sus siglas en inglés) para medir distancias cosmológicas en el Universo, y la hemos aplicado a una muestra de galaxias del Extended Baryon Oscillation Spectroscopic Survey (eBOSS).
\item Hemos utilizado software de cosmología avanzada como CLASS, RUSTICO y BRASS para desarrollar un pipeline que obtiene mediciones de distancia cosmológica de un catálogo de galaxias utilizando la técnica BAO.
\item Además, hemos desarrollado código utilizando la librería Matplotlib escrita en Python para generar visualizaciones de alta calidad de nuestros resultados.
\item Los ordenadores de alto rendimiento del grupo de investigación FQM-378 de la Universidad de Córdoba se han utilizado para calcular la Transformada Rápida de Fourier de la función de correlación de 2 puntos de las galaxias de eBOSS, y para determinar las posiciones de las características de BAO en el espectro de potencia, junto con sus incertidumbres.
\item Suponiendo un modelo cosmológico plano, nuestras mediciones de distancia inferidas a las galaxias de eBOSS (normalizadas a la escala del horizonte de sonido) son: $D_H/r_d = 18.68 \pm 0.72$ y $D_M/r_d = 18.13 \pm 0.58$ para la distancia angular, consistentes con trabajos previos~\cite{hector}.
\item Nuestro resultado principal es que no hay dependencia significativa de estas observables con respecto a cambios en el valor asumido del parámetro $ \Omega_k$ en el rango de estudio.
\end{enumerate}

Por lo tanto, concluimos que la suposición de un valor particular de $\Omega_k$ en la conversión de corrimiento al rojo a distancia no tiene ningún efecto significativo (al menos en el rango $\Omega_k \in [-0.20, +0.20]$) en las distancias cosmológicas inferidas a las galaxias de eBOSS utilizando la metodología BAO.

\chapter*{Conclusions}
\addcontentsline{toc}{chapter}{Conclusions}

Relative to the objectives stated in the chapter~\ref{cha:objectives}, and seeing the results obtained in the chapter~\ref{cha:results}, the following conclusions can be made from this work.

\begin{enumerate}
	\item We have reviewed and implemented the Baryon Acoustic Oscillation (BAO) methodology to measure cosmological distances in the Universe, and we have applied it to a sample of galaxies from the extended Baryon Oscillation Spectroscopic Survey (eBOSS)
	\item We have used advanced cosmology software such as CLASS, RUSTICO and BRASS to develop a pipeline that obtains cosmological distance measurements of a galaxy catalogue using the BAO technique.
	\item In addition, we have developed code using the Matplotlib library written in Python to generate quality level visualization of our results.
	\item The high performance computers of the University of Cordoba FQM-378 research group have been used to calculate the Fast Fourier Transform of the 2-point correlation function of eBOSS galaxies, and to determine the positions of the BAO features in the power spectrum, together with their uncertainties.
	\item Assuming a flat cosmological model, our inferred distance measurements to eBOSS galaxies (normalized to the sound horizon scale) are: $D_H/r_d = 18.68 \pm 0.72$ and $D_M/r_d = 18.13 \pm 0.58$ for the angular diameter distance, consistent with previous works~\cite{hector}.
	\item Our main result is that there is no significant dependence of these observables with respect to changes in the assumed value of the $\Omega_k$ parameter in the range of study.
\end{enumerate}

Therefore, we conclude that the assumption of a particular $\Omega_k$ value in the redshift-to-distance conversion does not have any significant effect (at least in the range $\Omega_k \in [-0,20, +0,20]$) on the inferred cosmological distances to eBOSS galaxies using the BAO methodology.

