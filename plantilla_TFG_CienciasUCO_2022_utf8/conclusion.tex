\chapter*{Conclusiones}
\addcontentsline{toc}{chapter}{Conclusiones}

En este trabajo se ha hecho uso de todas las herramientas mencionadas en el capítulo~\ref{cap:met-mat} para calcular la variación de los distintos observables con la curvatura del universo, y verificar así que todos los cálculos que se han hecho hasta ahora (que asumen un universo plano) son válidos. Se ha obtenido en la figura~\ref{fig:DA_DH} que efectivamente para $\Omega_k=0.00$ ambos parámetros  $\alpha_\parallel$ y $\alpha_\perp$ son compatibles con  $\alpha=1$, (debido a que se encuentran a menos de $3\sigma$, siendo $\sigma$ la varianza de cada $\alpha$\footnote{No se especifica si $\alpha_\parallel$ o $\alpha_\perp$ ya que se hace referencia a ambas cantidades a la vez. Lo mismo se hará en el siguiente párrafo con $D/r_s$, que hace referencia tanto a $D_H /r_s$ como a $D_A/r_s$.}.\\

Además, se puede también observar que la tendencia ascendente de las  $\alpha$ contrarresta la tendencia descendiente de $D /r_s$, lo que indica 	que variar $\Omega_k$ no afecta de manera significativa las mediciones de los observables, de hecho siendo todos estos $D /r_s$ compatibles los unos con los otros, ya que todas las barras de error se superponen.\\

A pesar de esto, se observa también cierta tendencia descendiente en $D_H /r_s$ para $\Omega_k<0$, mientras que $D_A / r_s$ se mantiene aproximadamente constante. Esto indica que probablemente sea más acertado escoger $\Omega_k \ge 0$, que $\Omega_k<0$. \\

Con esto se puede concluir, que aunque $\Omega_k=0.00$ es una buena elecciónpara el parámetro de curvatura del universo, se puede variar ligeramente este parámetro de forma que la densidad del universo varíe hasta en un 20\% con respecto de la densidad crítica.


\chapter*{Conclusions}
\addcontentsline{toc}{chapter}{Conclusions}

In this work ...
