\chapter{Methods and materials.}

For this work, many diferent tools were used, which will be cathegorized in hardware, software and mathematical tools. 

In terms of hardware, the author was granted access to the \textit{FQM-378} clusters in the Universidad de Córdoba. Access to these clusters was crucial for the calculations done throughout the work, reducing the time needed for each calculation in several orders of magnitude. Besides the clusters, the author also needed his own personal computer, mainly to remotely access the clusters and also for other types of calculations that could not have been done from the clusters. These calculations include among other things, plotting of figures. 

In terms of software, the main tools for this work were J. Lesgourges, T. Tram and N. Schoenberg's work, Cosmic Linear Anisotropy Solving System (CLASS), Hector Gil-Marín's works Rapid foUrier STatIstics COde (RUSTICO) and Bao and Rsd Algorithm for Spectroscopic Surveys (BRASS).


	
