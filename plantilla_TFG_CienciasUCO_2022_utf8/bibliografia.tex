%%%%%%%%%%%%%%%%%%%%%%%%%%%%%%%%%%%%%%%%%%%%%%%%%%%%%%%%%
%%%%  LA BIBLIOGRAFÍA %%%%%%%%%%%%%%%%%%%%%%%
%%%%%%%%%%%%%%%%%%%%%%%%%%%%%%%%%%%%%%%%%%%%%%%%%%%%%%%%%

\addcontentsline{toc}{chapter}{Bibliografía}
\begin{thebibliography}{999}
	\bibitem{Planck2018} Planck Collaboration. (2018). Planck 2018 results. VI\@. Cosmological parameters. \textit{Astronomy \& Astrophysics}, 641, A6. \url{https://doi.org/10.1051/0004-6361/201833910}

	\bibitem{Penzias1965} Penzias, A. A., \& Wilson, R. W. (1965). A Measurement of Excess Antenna Temperature at 4080 Mc/s. \textit{The Astrophysical Journal}, 142, 419. \url{https://doi.org/10.1086/148307}

\bibitem{SmootMather} Smoot, G. F., \& Mather, J. C. (1992). Structure in the COBE differential microwave radiometer first-year maps. \textit{The Astrophysical Journal}, 396, L1-L5. \url{https://doi.org/10.1086/186504}

\bibitem{2dFCole2005} Cole, S., et al. (The 2dFGRS Collaboration) (2005). The 2dF Galaxy Redshift Survey: power-spectrum analysis of the final dataset and cosmological implications. \textit{Monthly Notices of the Royal Astronomical Society}, 362, 505-534. \url{https://doi.org/10.1111/j.1365-2966.2005.09318.x}

\bibitem{Eisenstein2005} Eisenstein, D. J., et al. (The SDSS Collaboration) (2005). Detection of the Baryon Acoustic Peak in the Large-Scale Correlation Function of SDSS Luminous Red Galaxies. \textit{The Astrophysical Journal, 633, 560-574}. \url{https://doi.org/10.1086/466512}

\bibitem{Hubble1929} Hubble, E. (1929). A relation between distance and radial velocity among extra-galactic nebulae. \textit{Proceedings of the National Academy of Sciences}, 15(3), 168-173. \url{https://doi.org/10.1073/pnas.15.3.168}

\bibitem{class} Lesgourgues, J., Tram, T., \& Sprenger, T. The Cosmic Linear Anisotropy Solving System (CLASS) IV: efficient implementation of the Cosmic Microwave Background and large scale structure likelihoods. \textit{Journal of Cosmology and Astroparticle Physics}, 2011(07), 002. \url{https://github.com/lesgourg/class_public}.

\bibitem{montepython} Brinckmann, T. Montepython: Pythonic MCMC for Cosmology. \textit{Journal of Open Source Software}, 3(24), 676, 2018. \url{https://github.com/montepython/montepython_public}.

\bibitem{rustico} Gil-Marín, H. RUSTICO: A fast and scalable method for measuring the autocorrelation function of galaxy surveys. \textit{Astronomy and Computing}, 31, 100391, 2020. \url{https://github.com/hectorgil/rustico}.

\bibitem{brass} Gil-Marín, H. BRASS: BAO and RSD Analysis of Spectra with Systematics. \textit{Journal of Open Source Software}, 4(45), 1884, 2019. \url{https://github.com/hectorgil/Brass}.
\bibitem{hector}
Héctor Gil-Marín, et al.
\newblock The Completed {SDSS}-{IV} extended Baryon Oscillation Spectroscopic Survey: measurement of the {BAO} and growth rate of structure of the luminous red galaxy sample from the anisotropic power spectrum between redshifts 0.6 and 1.0.
\newblock \emph{Monthly Notices of the Royal Astronomical Society}, 498(2):2492--2531, Aug. 2020.
\newblock URL: \url{https://arxiv.org/abs/2007.08994}



\bibitem{eBoss}
S.~Alam \textit{et al.} [eBOSS],
%``Completed SDSS-IV extended Baryon Oscillation Spectroscopic Survey: Cosmological implications from two decades of spectroscopic surveys at the Apache Point Observatory,''
Phys. Rev. D \textbf{103}, no.8, 083533 (2021)
doi:10.1103/PhysRevD.103.083533
[arXiv:2007.08991 [astro-ph.CO]].
%588 citations counted in INSPIRE as of 19 Apr 2023

\end{thebibliography}
