\chapter*{Appendix: Example of data visualisation code with python \& matplotlib}
\addcontentsline{toc}{chapter}{Appendix: Example of data visualisation code with python \& matplotlib}

\renewcommand{\baselinestretch}{1}
\begin{lstlisting}[language=python]
#python 3.9.7
#Snippet to plot Fig. 4.1. The other figures were done
#in a similar fashion. Firstly, read the alpha values
#and their standard deviation (std) from 'logfiles'.
#Calculate the fiducial values of the observables through
#their definitions in the introduction. Knowing the alpha
#values and the fiducial observable, calculate the measured
#observables. Each step plots a row in the figure.
import numpy as np
import matplotlib.pyplot as plt
import scipy.constants as ct
import util_tools #Library with custom functions

#Wrapper to make certain function return 
#an iterable if input was iterable and 
#return scalar if input was scalar
def iterable_output(func):
    def wrapper(zmax, Ok):
        try:
            iter(Ok) #Cheks if Ok an array
			#Calculate by recursion the desired array
			result = [func(zmax, ok) for ok in Ok] 
            return np.array(result)
        except TypeError: #input was scalar
            return func(zmax, Ok)
    return wrapper

#Open the desired files
files = list(Path('../logfiles_phase2').glob('*3rd*')) 
#*3rd* as in third iteration, as explained in Chapter 3

Ok_list = [] #The list of the Omega_k to be used
a_para = []  #alpha_parallel list with 
		#each mean and standard deviation
a_perp = []  #alpha_perpendicular list with 
		#each mean and standard deviation

for data in files:
    out = util_tools.alpha_from_logfile(data) #Read from Brass output
    Ok_list.append(out[0])
    a_para.append(out[1])
    a_perp.append(out[2])

n_points = 500
Ok_min, Ok_max = min(Ok_list), max(Ok_list)
Ok_cont = np.linspace(Ok_min, Ok_max, n_points) #For plotting reasons
Ok_rang = Ok_max - Ok_min
def H(z, Ok, Om=0.31):
    return H0*np.sqrt(Om*(1+z)**3 + Ok*(1+z)**2 + 1-Ok-Om)

#Cosmological observables for a certain cosmology
@iterable_output #Wrapper above defined
def DH_fid(z, Ok):
    return ct.c/1000/H(z, Ok) #/1000 factor to match H
    			      #and c units
@iterable_output
def DC_fid(z, Ok):
    return sp.integrate.quad(DH_fid, 0, z, args=(Ok,))[0]
@iterable_output
def DM_fid(z, Ok):
    DC = DC_fid(z, Ok)
    DH = DH_fid(z, Ok)
    if Ok>0:
        k =  DH/np.sqrt(Ok)
        return k*np.sinh(np.sqrt(Ok)*DC/DH)
    elif Ok<0:
        k =  DH/np.sqrt(np.abs(Ok))
        return k*np.sin(np.sqrt(np.abs(Ok))*DC/DH)
    elif not Ok:
        return DC

fig, axes = plt.subplots(3, 2, sharex=True, figsize=(10, 7))

for Ok, apara, aperp in zip(Ok_list, a_para, a_perp):
    current_DH = DH_fid(zmax, Ok) * apara[0]/rs
    current_DH_std = DH_fid(zmax, Ok) * apara[1]/rs
    current_DM = DM_fid(zmax, Ok) * aperp[0]/rs
    current_DM_std = DM_fid(zmax, Ok) * aperp[1]/rs
    axes[0,0].errorbar(Ok, apara[0], yerr=apara[1], fmt='x')
    axes[0,1].errorbar(Ok, aperp[0], yerr=aperp[1], fmt='x')
    axes[2,0].errorbar(Ok, current_DH,  yerr=current_DH_std, fmt='x') 
    axes[2,1].errorbar(Ok, current_DM,  yerr=current_DM_std, fmt='x') 

axes[1,0].plot(Ok_cont, DH_fid(zmax, Ok_cont)/rs) 
axes[1,1].plot(Ok_cont, DM_fid(zmax, Ok_cont)/rs)
axes[0,0].set_ylabel(r'$\alpha_{\parallel}$')
axes[0,1].set_ylabel(r'$\alpha_{\perp}$')
axes[1,0].set_ylabel(r'$\left[ D_H/r_d\right]^{r}$')
axes[1,1].set_ylabel(r'$\left[ D_M/r_d\right]^{r}$')
axes[2,0].set_ylabel(r'$D_H/r_d$')
axes[2,1].set_ylabel(r'$D_M/r_d$')
axes[2,0].set_xlabel(r'$\left[ \Omega_k\right]^{r}$')
axes[2,1].set_xlabel(r'$\left[ \Omega_k\right]^{r}$')
axes[2,0].set_xticks(Ok_list)
axes[2,0].set_xticklabels(Ok_list)
axes[2,1].set_xticklabels(Ok_list)
plt.show()
\end{lstlisting}
\renewcommand{\baselinestretch}{1.5}



