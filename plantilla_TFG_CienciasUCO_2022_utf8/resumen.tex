\chapter*{Resumen}
\addcontentsline{toc}{chapter}{Resumen. Palabras clave}

En este trabajo de fin de grado se hace uso de herramientas de computación de alto rendimiento y análisis de datos para estudiar los efectos de ligeras variaciones en el modelo cosmológico estándar, el modelo $\Lambda$ \textit{Cold Dark Matter} ($\Lambda$CDM). Si bien este modelo asume un universo espacialmente plano, se observa que las variaciones de hasta un 20\% en el parámetro de curvatura del universo $\Omega_k$ no tienen consecuencias significativas en los observables que nos interesan.  \\

Este trabajo se basa en las oscilaciones acústicas de bariones, un fenómeno que nos permite estudiar el comportamiento del universo en sus etapas más tempranas (los primeros 380.000 años de sus 13.800 millones de años de vida, ¡un 0,03\% de la vida del universo!). Estas oscilaciones dan forma a la estructura a gran escala del universo y, lo que es más importante, establecen una "regla cósmica" $r_d$ con respecto a la cual se miden las distancias cosmológicas, como la distancia de Hubble $D_H$ y la distancia del diámetro angular $D_M$. \\

Después de analizar el catálogo de galaxias de la encuesta espectroscópica de oscilaciones acústicas de bariones extendida, se obtuvieron los siguientes resultados: $D_H/r_d = 18.66\pm 0.72$ y $D_M/r_d = 18.28\pm 0.53$ para un universo plano, en concordancia con los resultados para todos los valores del parámetro $\Omega_k$ indicados, y lo que es más importante, con resultados anteriores en el campo

\paragraph{Palabras clave:} Cosmología; Astrofísica; Oscilaciones Acústicas de Bariones; Análisis de datos.









\chapter*{Abstract}
\addcontentsline{toc}{chapter}{Abstract. Keywords}

In this Bachelor's Thesis we make use of high performance computing and data analysis tools to study the effects of slight variations in the Standard Cosmological model, the $\Lambda$ Cold Dark Matter ($\Lambda$CDM) model. While this model assumes a spatially flat universe, we observe that variations of around 20\% in the curvature parameter of the universe $\Omega_k$ have no significant consequences in the observables that interest us.\\

This work is based off the Baryon Acoustic Oscillations, a phenomenon that allows us to study the behaviour of the universe in its earliest stages (the first 380.000 of its 13.8 billion years of lifetime -- a 0.03\% of the Universe's lifetime!). These oscillations shape the large scale structure of the universe, and more importantly, set a `cosmic ruler' $r_d$ with respect to which is used to measure cosmological distances, such as the Hubble distance $D_H$ and angular diameter distance $D_M$.\\

After analyzing the extended Baryon Oscillation Spectroscopic Survey galaxy catalogue, we achieve the following results: $D_H/r_d = 18.66\pm 0.72$ y $D_M/r_d = 18.28\pm 0.53$ for a flat universe, in concordance to the results for all the stated  $\Omega_k$ parameter values, and more importantly with previous results in the field.





\paragraph{Keywords:} Cosmology; Astrophysics; Baryon Acoustic Oscillations; Data Analysis.
