\chapter*{Resumen}
\addcontentsline{toc}{chapter}{Resumen. Palabras clave}

Escriba aquí un resumen de la memoria en castellano que contenga entre 100 y 300 palabras. Las palabras clave serán entre 3 y 6.

\paragraph{Palabras clave:} palabra clave 1; palabra clave 2; palabra clave 3; palabra clave 4 









\chapter*{Abstract}
\addcontentsline{toc}{chapter}{Abstract. Keywords}


In this End of Degree Work, we make use of high performance computing and data analysis tools to study the effects of slight variations in the Standard Cosmological model, the $\Lambda$ Cold Dark Matter ($\Lambda$CDM) model. While this model assumes a spatially flat universe, we observe that variations of around 20\% in the curvature parameter of the universe $\Omega_k$ have no significant consequences in the observables that interest us.
This work is based off the Baryon Acoustic Oscillations, a phenomenon that allows us to study the behaviour of the universe in its earliest stages (the first 380.000 of its 13.8 billion years of lifetime -- a 0.03\% of the Universe's lifetime!). These oscillations dictate the large scale structure of the universe, and more importantly, set a `cosmic ruler' $r_d$ with respect to which is used to measure cosmological distances, such as the Hubble distance $D_H$ and angular diameter distance $D_M$.
After using advanced data analysis tools, data visualisation tools and high performance computer clusters, we achieve the following results: $D_H/r_d = XX\pm$ and $D_M /r_d = XX \pm XX$ for a flat universe, in concordance to previous results in the field. 





\paragraph{Keywords:} keyword1; keyword2; keyword3; keyword4
